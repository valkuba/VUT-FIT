\documentclass[twocolumn, a4paper, 11pt]{article}
\usepackage[utf8]{inputenc}
\usepackage[czech]{babel}
\usepackage[IL2]{fontenc}
\usepackage{times}
\usepackage{amsthm}
\usepackage{amsmath}
\usepackage{amsfonts}

\newtheorem{Def}{Definice}
\newtheorem{Veta}{Věta}

\usepackage[top=2.5cm, left=1.5cm, text={18cm, 25cm}]{geometry}

\title{proj2}
\author{Jakub Valeš }
\date{March 2022}

\begin{document}
    \begin{titlepage}
        \begin{center}
            \textsc
            {\Huge{Vysoké učení technické v~Brně}\\[0.4em]
            {\huge Fakulta informačních technologií}}\\[0.3em]
            \vspace{\stretch{0.382}}
            {\huge Typografie a~publikování --\ 2. projekt\\
            Sazba dokumentů a~matematických výrazů}\\
            \vspace{\stretch{0.618}}
        \end{center}
    {\LARGE 2022 \hfill Jakub Valeš (xvales04)}
    \end{titlepage}
    
\section*{Úvod}
    V~této úloze si vyzkoušíme sazbu titulní strany, matematických vzorců, prostředí a~dalších textových struktur obvyklých pro technicky zaměřené texty (například rovnice (\ref{Rovnice2}) nebo Definice \ref{Def2} na straně \pageref{Def2}). Pro vytvoření těchto odkazů používáme příkazy \verb|\label|, \verb|\ref| a~\verb|\pageref|.
    
    Na titulní straně je využito sázení nadpisu podle optického středu s~využitím zlatého řezu. Tento postup byl probírán na přednášce. Dále je na titulní straně použito odřádkování se zadanou relativní velikostí 0,4 em a~0,3 em.


\section{Matematický text}
    Nejprve se podíváme na sázení matematických symbolů a~výrazů v~plynulém textu včetně sazby definic a~vět s~využitím balíku \texttt{amsthm}. Rovněž použijeme poznámku pod čarou s~použitím příkazu \verb|\footnote|. Někdy je vhodné použít konstrukci \verb|${}$| nebo \verb|\mbox{}|, která říká, že (matematický) text nemá být zalomen. 
    
    \begin{Def}
        \textup{Nedeterministický Turingův stroj} (NTS) je šestice tvaru ${M=(Q,\Sigma,\Gamma,\delta,q_0,q_F)}$, kde:
    \begin{itemize}
        \item Q je konečná množina \textup{vnitřních (řídicích) stavů},
        
        \item ${\Sigma}$ je konečná množina symbolů nazývaná \textup{vstupní abeceda}, ${\Delta \notin \Sigma}$
        
        \item ${\Gamma}$ je konečná množina symbolů, ${\Sigma \; \subset \; \Gamma, \; \Delta \; \in \; \Gamma}$ nazývaná \textup{pásková abeceda},
        
        \item ${\delta : (Q\:\backslash\:\{q_F\}) \times \Gamma \rightarrow 2^{Q\times(\Gamma \cup \{L,R\})}}$, kde L, R ${\notin \Gamma}$, je parciální \textup{přechodová funkce}, a
        
        \item ${q_0 \in Q}$ je \textup{počáteční stav} a~${q_F \in Q}$ je \textup{koncový stav}.
    \end{itemize}
    \end{Def}
    
    Symbol ${\Delta}$ značí tzv. \emph{blank} (prázdný symbol), který se vyskytuje na místech pásky, která nebyla ještě použita.
    
    \emph{Konfigurace pásky} se skládá z~nekonečného řetězce, který reprezentuje obsah pásky, a~pozice hlavy na tomto řetězci. Jedná se o~prvek množiny ${\{\gamma \Delta^\omega \:|\: \gamma \in \Gamma^*\} \times \mathbb{N}}$\footnote{Pro libovolnou abecedu ${\Sigma}$ je ${\Sigma^\omega}$ množina všech nekonečných řetězců nad $\Sigma$, tj. nekonečných posloupností symbolů ze ${\Sigma}$.}.
    \emph{Konfiguraci pásky} obvykle zapisujeme jako${\Delta xyz\underline{z}x\Delta \dots}$ (podtržení značí pozici hlavy).
    \emph{Konfigurace stroje} je pak dána stavem řízení a~konfigurací pásky. Formálně se jedná o~prvek množiny ${Q \times \{\gamma \Delta^\omega \:|\: \gamma \in \Gamma^*\} \times \mathbb{N}}$.
    
    \subsection{Podsekce obsahující definici a~větu}
        \begin{Def}
        \label{Def2}
            \textup{Řetězec $w$ nad abecedou ${\Sigma}$ je přijat NTS~M}, jestliže M při aktivaci z~počáteční konfigurace pásky ${\underline{\Delta} w \Delta \dots}$ a~počátečního stavu ${q_0}$ může zastavit přechodem do koncového stavu ${q_F}$, tj. $(q_0,\Delta w \Delta^\omega,0) \overset{*} {\underset{M}{\vdash}} (q_F,\gamma,n) $ pro nějaké ${\gamma \in \Gamma^* a~n \in \mathbb{N}}$.
            
            Množinu $L(M) = \{w\;|\;w $ je přijat NTS $M \} \subseteq \Sigma^*$ \textup{nazýváme jazyk přijímaný NTS} $M$. 
        \end{Def}
        Nyní si vyzkoušíme sazbu vět a~důkazů opět s~použitím balíku \texttt{amsthm}.
        \begin{Veta}
            Třída jazyků, které jsou přijímány NTS, odpovídá \textup{rekurzivně vyčíslitelným jazykům}.
        \end{Veta}
        
        
\section{Rovnice}
    Složitější matematické formulace sázíme mimo plynulý text. Lze umístit několik výrazů na jeden řádek, ale pak je třeba tyto vhodně oddělit, například příkazem \verb|\quad|.
    
    $${x^2 - \sqrt[4]{y_1 * y_2^3} \quad x > y_1 > y_2 \quad z_{z_z} \neq a_1^{{a_2}^{a_3}}}$$
    
    V~rovnici (\ref{Rovnice1}) jsou využity tři typy závorek s~různou explicitně definovanou velikostí.
        
        \begin{eqnarray}
            x & = & \Bigg\{ a~\oplus \bigg[ b \cdot \big(c \ominus d \big) \bigg] \Bigg\}^{4/2} 
        \label{Rovnice1}
            \\
            y & = & \lim\limits_{\beta\rightarrow\infty} \frac{\tan^2 \beta - \sin^3     \beta}{\frac{1}{\frac{1}{\log_{42}x}+\frac{1}{2}}}
        \label{Rovnice2}
        \end{eqnarray}
        

    V~této větě vidíme, jak vypadá implicitní vysázení limity $ \textup{lim}_{\beta\rightarrow\infty} f(n)$ v~normálním odstavci textu. Podobně je to i~s~dalšími symboly jako $\bigcup_{N \in \mathcal{M}}$ či $\Sigma^n_{j = 0} x^2_j$. 
    S~vynucením méně úsporné sazby příkazem \verb|\limits| budou vzorce vysázeny v~podobě 
    $\lim\limits_{\beta\rightarrow\infty} f(n)$ a~$\sum\limits_{j=0}^n x_j^2$.
    
    

\section{Matice}
Pro sázení matic se velmi často používá prostředí \texttt{array} a~závorky (\verb|\left|, \verb|\right|). 
    $$ A = \left| \begin{array}{cccc}
    a_{11} & a_{12} & \ldots & a_{1n} \\
    a_{21} & a_{22} & \ldots & a_{2n} \\
    \vdots & \vdots & \ddots & \vdots \\
    a_{m1} & a_{m1} & \ldots & a_{mn} 
    \end{array} \right| 
    = \left| \begin{array}{cc}
    t & u \\
    v & w
    \end{array} \right|
    = tw - uv $$
Prostředí \texttt{array} lze úspěšně využít i~jinde.
    $$\binom{n}{k}
    = \left\{ \begin{array}{cl}
    \frac{n!}{k!(n-k)!} & \text{pro } 0 \leq k \leq n \\
    0 & \text{pro } k > n \text{ nebo } k < 0
    \end{array} \right.
    $$



    
\end{document}
